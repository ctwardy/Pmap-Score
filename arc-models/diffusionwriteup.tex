\documentclass[]{article}

%opening
\title{Modeling Lost Person Movement as Diffusion Using Elevation and Land Cover}
\author{Eric Cawi}

\begin{document}

\maketitle

\begin{abstract}
This work creates a model for lost person movement based on gas diffusion. to be filled in
\end{abstract}
\tableofcontents

We use Burgess and Darken's \ycite{burgess_realistic_2004} gas diffusion model of path planning to create probability maps for lost hikers. Where Burgess and Darken sample the diffusion gradient to find discrete likely paths from source to target, we use the diffusion gradient itself as a probability map, and find it is competitive with the Euclidean ring model in \cite{koester_lost_2008}.

\section{Create Working Gas Model}

\textbf{Insert diagram here}. 

First, the area is divided into cells. Each cell has a concentrationThen the elevation data is transformed onto the interval $[0,.25]$, which represents the maximum transfer of gas from that cell to the another cell. For example, a cell with a value of .25 and a concentration of 1 (units) can transfer up to .25 of its gas to each of the four neighboring cells. This ensures the conservation of concentration in the simulation. This simulation also allows for sources and sinks to be placed at arbitrary grid cells. Burgess and Darken run this simulation until a relative equilibrium is reached (mathematically taking the limit as time goes to infinity), and then use the gradient of the final concentration to generate possible paths of approach for military applications. Note-using the gradient to generate the avenues of approach is solving the transport equation-hard to solve numerically.

For search and rescue, I am more interested in the final gas concentration. Modeling the last known point as a source, the gas model simulates the overall movement of many hikers from the last known point (assertion, want to prove this). I use the source because it models a spike of probability at the LKP, and as B/D used originally it represents a continuous stream of people moving around the material

In this model, I use elevation/land cover data as the speed at which the gas is allowed to propagate. Because the arrays involved are very large, I downsample by a factor of 4 on the NED dataset, then resample up to 5001x5001 pixels using Matlab's default interpolation- i think it's bicubic. Currently I'm running out to a huge number of iterations because I haven't calibrated the transformations to match different categories/mean distances yet.
\section{Transformations from elevation/land cover to impedance}
Elevation- using Tobler's hiking function
Land Cover - using 1-Don's values
\section{Diffusion Equation with source}
The diffusion equation models the overall random walks of particles diffusing through materials. At each point there is some diffusion constant determining the maximum velocity of the concentration. Mathematically, freely diffusing gas like the Burgess Darken model can be modeled with this equation. In 1-D with a constant diffusion coefficient, the solution is a normal distribution, and generalizations give similarly smooth distributions. In our case the diffusion is given by:
\[\frac{dc(x,y,t)}{dt}=\nabla \cdot(D(x,y)\nabla c(x,y,t)=\]
\[(\frac{dD(x,y)}{dx},\frac{dD(x,y)}{dy})\cdot\nabla c(x,y,t) + D(x,y)\Delta c(x,y,t)\]
Current work: finite difference scheme for this equation.
\section{Boundary Considerations}
I'm considering using the boundary of the map as a sink and seeing what happens, or having some other way of modeling the interactions with the outside layer because in the current state we might be getting innaccurate reflection. Boundary conditions will be important for the PDE formulation as well.

\section{Static Model Preliminary Testing}
Done on 10 arizona cases I had elevation data for. Observations- On flat terrain the concentration behaves like regular diffusion, i.e. a smooth version of the distance rings. Otherwise, it avoids big changes in elevation, as expected. E.g. if a subject doesn't start on a mountain then the mountain will have lower probability. However, a 2 or 3 of the hikers tested were going for the mountain, which makes sense because they were probably trying to hike on the mountain, but this model isn't going to catch that. Fixing this would require more information about the subject, for example "They were heading towards Mt. Awesome when they got lost." Since the odds are they moved in the direction of Mt. Awesome, a sink dropped at the intended location would skew the concentration in that direction. But that isn't available all the time.
Not done for land cover yet
\section{Scaling with Bob's Book}
not done
\section{Testing/fitting on Yosemite Cases}
Prediction- lots of mountains in yosemite, so we should see higher probabilities around ridges/trails that stay at the same elevation. I would like to somehow set up this as a batch

\section{Comparison with diffusion Models}
Compare results and speed.
\section{Discussion}
Goal: An easy GIS ready application/A diffusion model that responds to terrain and is about as good as the distance model, which means that the search team now has more information.


\section{Bacteria Model}
\textbf{Justin Spiegel}

This section discusses a motion-model variation inspired by the chemotaxis of \emph{Eschericha Coli} (\emph{E. Coli}) bacteria that reside in the lower intestine of mammals. \emph{E. Coli} have molecular 'taste-buds' that allow them to remember the level of glucose in their environment one second into the past. These taste-buds then power a flagellum, which receives from them a conditional command:
\begin{verbatim}
if glucose gradient > T:
    move forward
else:
    change direction at random
\end{verbatim}
Such models are \emph{myopic} and \emph{stubborn}.  They are myopic in that they choose based only on the local gradient, rather than solving field equations. They are stubborn in that they prefer the current direction to to potentially more rewarding alternatives, so long as the current direction is acceptable, meaning that it exceeds a certain threshold of desirability. Stubbornness may model well-known human tendencies to satisfice rather than optimize.  

Learning behavior for the \emph{E. coli} model would be relatively easy: simply take the difference in a continuous feature between end and start location, divide it by some estimation or calculation of the time, and multiply it by some constant $K$ to get the minimum threshold that the agent's path would need to exceed for the direction to remain unchanged.

\section{References}
\bibliographystyle{plain}
\bibliography{diffusionwriteup}

%Need Citations
\end{document}
